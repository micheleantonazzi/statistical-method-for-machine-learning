\section{Models}\label{header-n90}

This section reports the architecture and hyper-parameters of the
neural networks used to classy the images. In particular, the model
tested are feed-forward neural networks (FFNNs) and convolutional neural
networks (CNNs).

\subsection{Feed-forward neural network}\label{header-n92}

The feed-forward neural network is the first and simplest type of
artificial neural network. The information moves in only one direction,
forward, from the input nodes, through the hidden nodes, and to the
output nodes. The layers of neurons are fully connected, in the sense
that each neuron of a layer is connected to all neurons of the previous
and next layers. The following tables specify the architecture and the
hyper-parameters for each FFNN tested in the experiments. In particular, for each layer, the
number of neurons (called \emph{Units}) and the activation function are
reported. Instead, for the hyper-parameters, the tables contain the
weight estimator and its learning rate, the loss function, the number of
epochs, and the batch size.

The first model (called \emph{Perceptron}) is a classical neural network
with the simplest architecture: the single layer perceptron. It consists
of a single layer which is also the output layer. Its purpose is to
examine the network performance with the given dataset to build better
models.

\begin{longtable}[]{@{}llll@{}}
	\toprule
	\textbf{Layers} & \textbf{Type} & \textbf{Units} & \textbf{Activation}\tabularnewline
	\midrule
	\endhead
	Layer 1 & Input & - & -\tabularnewline
	Layer 2 & Flatten & - & -\tabularnewline
	Layer 3 & Dense & 10 & Linear\tabularnewline
	\bottomrule
	\caption{\emph{Perceptron} architecture}
\end{longtable}

\begin{longtable}[]{@{}ll@{}}
	\toprule
	\textbf{Parameter} & \textbf{Value}\tabularnewline
	\midrule
	\endhead
	Weight estimator & adam\tabularnewline
	Learning rate & 0.001\tabularnewline
	Loss function & sparse categorical crossentropy\tabularnewline
	Epochs & 20\tabularnewline
	Batch size & 256\tabularnewline
	\bottomrule
	\caption{\emph{Perceptron} hyper-parameters}
\end{longtable}

\newpage

The second model is called \emph{FFNN}. It is similar to the first with
more complex architecture. In particular, this network has two hidden
layers.

\begin{longtable}[]{@{}llll@{}}
	\toprule
	\textbf{Layers} & \textbf{Type} & \textbf{Units} & \textbf{Activation}\tabularnewline
	\midrule
	\endhead
	Layer 1 & Input & - & -\tabularnewline
	Layer 2 & Flatten & - & -\tabularnewline
	Layer 3 & Dense & 64 & ReLU\tabularnewline
	Layer 4 & Dense & 32 & ReLU\tabularnewline
	Layer 5 & Dense & 10 & Linear\tabularnewline
	\bottomrule
	\caption{\emph{FFNN} architecture}
\end{longtable}

\begin{longtable}[]{@{}ll@{}}
	\toprule
	\textbf{Parameter} & \textbf{Value}\tabularnewline
	\midrule
	\endhead
	Weight estimator & adam\tabularnewline
	Learning rate & 0.001\tabularnewline
	Loss function & sparse categorical crossentropy\tabularnewline
	Epochs & 20\tabularnewline
	Batch size & 256\tabularnewline
	\bottomrule
	\caption{\emph{FFNN} hyper-parameters}
\end{longtable}

\subsection{Convolutional neural networks}\label{header-n186}

The next models are convolutional neural networks. A CNN at first
learns what are the data features using convolutional layers and
subsequently uses these features to label the data using fully
connected layers. CNNs are often used to analyze visual images thanks to
their space invariant characteristics. The following tables specify the
architecture and the hyper-parameters for each convolutional network tested in the experiments. In particular, for
each layer, the number of filters, the kernel size, and the
activation function are reported. The hyper-parameters are
the same as those listed in the previous section.
\newpage
The first convolutional neural network is called \emph{CNN\_1}. Its
architecture is really simple, with a single convolutional layer (after
the input layer), a pooling layer (which implements the max operation),
and a dense layer (before the output layer).

\begin{longtable}[]{@{}lllll@{}}
	\toprule
	\textbf{Layers} & \textbf{Type} & \textbf{Filters} & \textbf{Kernel size} &
	\textbf{Activation}\tabularnewline
	\midrule
	\endhead
	Layer 1 & Input & - & - & -\tabularnewline
	Layer 2 & Conv2D & 2 & 3, 3 & ReLU\tabularnewline
	Layer 3 & MaxPooling2D & - & 2, 2 & -\tabularnewline
	Layer 4 & Flatten & - & - & -\tabularnewline
	Layer 5 & Dense & 64 & & ReLU\tabularnewline
	Layer 6 & Dense & 10 & & Linear\tabularnewline
	\bottomrule
		\caption{\emph{CNN\_1} architecture}
\end{longtable}

\begin{longtable}[]{@{}ll@{}}
	\toprule
	\textbf{Parameter} & \textbf{Value}\tabularnewline
	\midrule
	\endhead
	Weight estimator & adam\tabularnewline
	Learning rate & 0.001\tabularnewline
	Loss function & sparse categorical crossentropy\tabularnewline
	Epochs & 20\tabularnewline
	Batch size & 256\tabularnewline
	\bottomrule
	\caption{\emph{CNN\_1} hyper-parameters}
\end{longtable}

The second convolutional neural network (called \emph{CNN\_2}) is
similar to \emph{CNN\_1} but it has another pair of convolutional and
pooling layer.

\begin{longtable}[]{@{}lllll@{}}
	\toprule
	\textbf{Layers} & Type & \textbf{Filters} & \textbf{Kernel size} &
	\textbf{Activation}\tabularnewline
	\midrule
	\endhead
	Layer 1 & Input & - & - & -\tabularnewline
	Layer 2 & Conv2D & 4 & 3, 3 & ReLU\tabularnewline
	Layer 3 & MaxPooling2D & - & 2, 2 & -\tabularnewline
	Layer 4 & Conv2D & 2 & 3, 3 & ReLU\tabularnewline
	Layer 5 & MaxPooling2D & - & 2, 2 & -\tabularnewline
	Layer 6 & Flatten & - & - & -\tabularnewline
	Layer 7 & Dense & 64 & & ReLU\tabularnewline
	Layer 8 & Dense & 10 & & Linear\tabularnewline
	\bottomrule
		\caption{\emph{CNN\_2} architecture}
\end{longtable}

\newpage

\begin{longtable}[]{@{}ll@{}}
	\toprule
	\textbf{Parameter} & \textbf{Value}\tabularnewline
	\midrule
	\endhead
	Weight estimator & adam\tabularnewline
	Learning rate & 0.001\tabularnewline
	Loss function & sparse categorical crossentropy\tabularnewline
	Epochs & 20\tabularnewline
	Batch size & 256\tabularnewline
	\bottomrule
		\caption{\emph{CNN\_2} hyper-parameters}
\end{longtable}

The third model (called \emph{CNN\_3}) is more complex than the previous
one in its architecture. It has another pair of convolutional and
pooling layer.

\begin{longtable}[]{@{}lllll@{}}
	\toprule
	\textbf{Layers} & \textbf{Type} & \textbf{Filters} & \textbf{Kernel size} &
	\textbf{Activation}\tabularnewline
	\midrule
	\endhead
	Layer 1 & Input & - & - & -\tabularnewline
	Layer 2 & Conv2D & 8 & 3, 3 & ReLU\tabularnewline
	Layer 3 & MaxPooling & - & 2, 2 & -\tabularnewline
	Layer 4 & Conv2D & 4 & 3, 3 & ReLU\tabularnewline
	Layer 5 & MaxPooling2D & - & 2, 2 & -\tabularnewline
	Layer 6 & Conv2D & 2 & 3, 3 & ReLU\tabularnewline
	Layer 7 & MaxPooling2D & - & 2, 2 & -\tabularnewline
	Layer 8 & Flatten & - & - & -\tabularnewline
	Layer 9 & Dense & 64 & & ReLU\tabularnewline
	Layer 10 & Dense & 10 & & Linear\tabularnewline
	\bottomrule
		\caption{\emph{CNN\_3} architecture}
\end{longtable}

\begin{longtable}[]{@{}ll@{}}
	\toprule
	\textbf{Parameter} & \textbf{Value}\tabularnewline
	\midrule
	\endhead
	Weight estimator & adam\tabularnewline
	Learning rate & 0.001\tabularnewline
	Loss function & sparse categorical crossentropy\tabularnewline
	Epochs & 20\tabularnewline
	Batch size & 256\tabularnewline
	\bottomrule
		\caption{\emph{CNN\_3} hyper-parameters}
\end{longtable}

\newpage

The last convolutional neural networks is called \emph{CNN\_4}. It has
the same architecture that the \emph{CNN\_3} but its hyper-parameters
are different. In particular, the number of epochs rises to 50 and the
batch size goes down to 64.

\begin{longtable}[]{@{}lllll@{}}
	\toprule
	\textbf{Layers} & \textbf{Type} & \textbf{Filters} & \textbf{Kernel size} &
	\textbf{Activation}\tabularnewline
	\midrule
	\endhead
	Layer 1 & Input & - & - & -\tabularnewline
	Layer 2 & Conv2D & 8 & 3, 3 & ReLU\tabularnewline
	Layer 3 & MaxPooling & - & 2, 2 & -\tabularnewline
	Layer 4 & Conv2D & 4 & 3, 3 & ReLU\tabularnewline
	Layer 5 & MaxPooling2D & - & 2, 2 & -\tabularnewline
	Layer 6 & Conv2D & 2 & 3, 3 & ReLU\tabularnewline
	Layer 7 & MaxPooling2D & - & 2, 2 & -\tabularnewline
	Layer 8 & Flatten & - & - & -\tabularnewline
	Layer 9 & Dense & 64 & & ReLU\tabularnewline
	Layer 10 & Dense & 10 & & Linear\tabularnewline
	\bottomrule
		\caption{\emph{CNN\_4} architecture}
\end{longtable}

\begin{longtable}[]{@{}ll@{}}
	\toprule
	\textbf{Parameter} & \textbf{Value}\tabularnewline
	\midrule
	\endhead
	Weight estimator & adam\tabularnewline
	Learning rate & 0.001\tabularnewline
	Loss function & sparse categorical crossentropy\tabularnewline
	Epochs & 50\tabularnewline
	Batch size & 64\tabularnewline
	\bottomrule
		\caption{\emph{CNN\_4} hyper-parameters}
\end{longtable}