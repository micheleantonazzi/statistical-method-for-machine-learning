\section{Introduction}
In the real world, there are a lot of different tasks that are too
complicated to be modeled by a conventional algorithm. Some problems
indeed may have a wide amount of data difficult to analyze. In this
case, build a specific algorithm means to understand the complex
patterns and the hidden correlations between the data. Instead, other
tasks may be influenced by a lot of external factors that generate a
large quantity of similar but different data. These factors are not easy
to model, especially considered all together, and often they are not a
priori known. This means that an algorithm performs well only in a
controlled environment, that respects specific preconditions. On the
other hand, if it is applied in the real world, the algorithm may
encounter data that it cannot correctly analyze. A particular field of
Computer Science is suitable to solve these situations:
machine learning (ML). It represents a family of algorithms that learn
automatically through experience. These algorithms are not designed for
a specific task but they are general purpose, so they can be used to
solve each type of task. The principle behind machine learning is the
following: each real phenomenon can be modeled as an unknown
mathematical function which can be approximate by a machine learning
algorithm. In particular, they build a mathematical model based on
sample data, known as training data, to make decisions or predictions
without being explicitly programmed to do so. This means that the data
play a central role in machine learning: they must be able to correctly
define the model behind the task. First of all, they must be sufficient
in number to generalize the problem, especially if the data have high
dimensionality. Secondly, they must be well-formed, in terms of the
range of values, scale, and distribution. Often a preprocessing
procedure is necessary to modify the data before being used by a
learning machine to improve its performance. There are three main
approaches to machine learning, depending on the nature and the type of
data available to a learning machine. The first is called
\emph{supervised learning} and consists in presenting to the learning
model the inputs with the correct outputs. The goal is to learn a
general function that maps inputs to outputs. Another machine learning
technique is \emph{unsupervised learning} where the input is not
associated with labels, leaving to the learning machine the task of
finding the data structure. Discovering the hidden patterns of data can
be a goal itself or the purpose can be the generation of new data with
similar characteristics. The last category is called \emph{reinforcement
learning} and occurs when a computer program interacts with a real
environment in which it must perform a certain task. The leaning machine
is provided by feedback which is analogous to rewards and it tries to
maximize them, as it navigates the specific problem space. Machine
learning algorithms are often used in computer vision tasks, like object
classification, object detection, motion analysis, and many others. In
the last few years, the state of the art methods to perform object
classification use machine learning methods, in particular \emph{deep learning.}
Deep learning is based on artificial neural networks, inspired by the
biological neural network that composed the animal brains. Specifically
to computer vision, the convolutional neural networks (CNNs) are
particularly suitable for analyzing images. They are able to learn how
to extract the features using the convolutional layers and,
subsequently, use these features in a space invariant way to better
generalize the data. In this work, it is performed a
multi-classification task over images using deep
learning models. The aim is to study the neural networks performances
based on their type, their architecture, and their hyper-parameters. In
particular, the model tested is the feed-forward neural network (FFNN)
and the convolutional neural network (CNN).